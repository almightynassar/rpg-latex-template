\documentclass[10pt,twoside]{article}
\usepackage{dnd}
\usepackage[utf8]{inputenc}
\usepackage{multicol}
\raggedcolumns
% Clickable table of content links
\usepackage[hidelinks]{hyperref}

% Start document
\begin{document}
\section*{Traveller-style D\&D Character Creation}
\addtocounter{section}{1}

Old-school gamers may remember the character creation rules for \textit{Traveller}, a science-fiction tabletop RPG. It was a mini-game where you you explored the history and background of your character (and in some versions your character could even die before the real game starts!). Despite some wonky mechanics, this mini-game fostered a deep player-character connection that is till unrivalled in the tabletop RPG world.

These rules are an adaption of \textit{Traveller} character generation for use in \textit{D\&D 5th edition}. While we have tried to balance these rules to make them compatible, this method will drastically alter your game's balance. You must seek the express permission from your Game Master before you use these rules in your game.

We try to stick to original and open content where possible; if you believe we have infringed on someone's copyright please contact us through the details at the end of this document!

\begin{multicols}{2}
\tableofcontents
\columnbreak

% =================
% Your content goes here
% =================

\section{Basic Concepts}

These rules completely rewrites the underlying character creation concepts in D\&D 5E, so some terminology will be different to what you are used to. Here is a quick summary of those changes:

\begin{itemize}
\item \textbf{Races} now only refer to the inherent genetic and physiological traits of your character. The standard Races you can find in the official Player's Handbook have been stripped of any learned cultural traits or skills.
\item \textbf{Upbringing} refers to the environment and culture that your character experienced during their childhood. Some Upbringings will contain the cultural traits and skills that have been stripped from the official Races in the Player's Handbook, while others are inspired from the official Backgrounds.
\item \textbf{Careers} is the direction that your character took once they reached adulthood for their race. This is where we will discover who your character is and what they achieved before they became an adventurer.
\item \textbf{Term} is the standard length of time for a career period. This varies between the different races, as stated under the Aging entry listed for your race. Longer-lived races have longer term lengths because they do not live at the frantic pace of humans, and will take their time exploring and perfecting their skills.
\item \textbf{Connections} are your character's links to this world. These links can be a \textit{Place}, \textit{Rival} or a \textit{Contact}. While character generation will prompt you to make (or advance) Connections, you can create your own Connections at any time with your Game Master's approval.
\item \textbf{Events} are significant moments in your character’s history. During your career you may be required to roll on a variety of different event tables, including:
\begin{itemize}
\item \textit{Life} These are general life events that involve your character such as marriages, births, funerals and friendships.
\item \textit{Career} These events significantly alter your working life, and could either harm or progress your career.
\item \textit{Aging} When your character reaches old age for their race, they are required to make a roll to see if they avoid any negative effects. When your character reaches the Advanced Aging limit your character has a significant risk of dying!
\item \textit{Mishaps} Life is never easy, and Mishaps represent the low points of your character's life.
\end{itemize}
\end{itemize}

Please keep in mind that everything is totally and utterly at the discretion of your Game Master. These rules have the potential to make game-breaking or otherwise unsuitable characters. The intention of these rules is to create a fun little mini-game out of character creation.

\textbf{And remember to Have fun!}

\section{Getting Started}

\begin{paperbox}{Skill Checks and Character Creation}
Your character starts their journey without any levels, and definitely without any proficiency. All Skill Checks that are requested during character creation (such as career, event, aging and mishap rolls) do not gain the proficiency bonus or any class traits that affect Skill Checks.
Any Race, Upbringing or Career traits that affect Skill Checks can be used during character creation.
\end{paperbox}

These rules completely modifies the Character Generation process given in the Player's Handbook, including the Race and Background entries. These rules will generate a character that is fully prepared to take their first level in a class.

\begin{enumerate}
\item Roll your six character ability scores or take the standard array, as per the rules on page 12 of the Player’s Handbook.
\item Choose a \textbf{Race} and add the traits to your character sheet. Your character’s age will be set to the Starting Age for your race.
\item Determine final Ability Scores and associated Modifiers.
\item Choose an \textbf{Upbringing} and add the traits to your character sheet.
\item Choose a \textbf{Career}. You cannot choose a career that you have already left.
\begin{itemize}
\item Roll to a skill check to qualify for that career
\item If you qualify, go to step 6
\item If you do not qualify, then you can enter the \textit{Serf} or the \textit{Vagabond} careers. If your Game Master allows it and it makes sense for your setting, you can be \textit{Drafted}.
\end{itemize}
\item Take the first \textbf{Rank} level for this career.
\begin{itemize}
\item If this is your first time in a career, add the Basic Training to your character sheet.
\item You do not gain the Basic Training benefit for subsequent careers. Basic Training for a career is a special trait that influences your character throughout their life, and follows them even in subsequent careers. However, the Game Master may allow you to replace your original Basic Training if it suits your character better.
\end{itemize}
\item Choose a \textbf{Specialisation} for this career and note it down on your character sheet. You may change specialisations during your career, keeping your Rank level as you do so.
\item Roll on the \textbf{Skills and Training} tables for your specialisation and add the traits to your character sheet.
\begin{itemize}
\item Some Game Masters may allow you to choose from the table instead of rolling for it.
\item Depending on your DM, one of three things can happen if you roll a skill or tool that you are already proficient in:
\begin{itemize}
\item Nothing; you lucked out
\item You get to reroll on the table
\item You can use the optional Advanced skills, Advanced languages or Advanced tools rules listed at the end of this document.
\end{itemize}
\end{itemize}
\item Roll a skill check for \textbf{Survival} in this career.
\begin{itemize}
\item If you succeed, go to Step 10
\item If you did not succeed, then events have forced you from this career. Roll on the Mishaps table, then go straight to Step 12.
\end{itemize}
\item Roll for \textbf{Events} on the career table. Alternatively, establish a Connection with another player character
\item Roll a skill check for \textbf{Advancement} in this career.
\begin{itemize}
\item If you succeed, increase your Rank and take any bonuses.
\item If you roll less than 8 + the number of terms you have spent in this career, you have been made redundant and must leave this career.
\end{itemize}
\item Increase your age according to the Term length stated in your race entry. If your character is over the age limits, roll skill check for \textbf{Aging}.
\item Choose whether to continue or change your career.
\begin{itemize}
\item If you continue with your career, go to Step 7
\item If you wish to finish this career, or if you were forced out, go to Step 14
\end{itemize}
\item Roll for textbf{Benefits}. You gain 1 dice for each term your spend in this career. You can also gain or lose Benefit rolls during events.
\begin{itemize}
\item If you want to choose another career go to Step 5
\item If you want to finish your character, go to Step 15
\end{itemize}
\item Finalise your character.
\begin{itemize}
\item Detail any Connection
\item Define the effects of Events and Mishaps
\item Purchase your starting equipment
\end{itemize}
\item Your character may now take their first level in a Class. Follow the rules for your Class as stated in the Players Handbook.
\end{enumerate}

\columnbreak

\section{Races}

\subsection{Dwarf}

You are a short, stout and extremely hardy humanoid. Many dwarves make their homes in the earth and under mountains.

\subsubsection*{Size}
You are 4-5 feet tall (1.2-1.5 metres) and weigh an average of 150 pounds (70 kg). Your size is medium.


\subsection{Elf}


\subsection{Halfling}


\subsection{Human}


\subsection{Dragonborn}


\subsection{Gnome}


\subsection{Half-Elf}


\subsection{Half-Orc}


\subsection{Tiefling}



\columnbreak

\begin{commentbox}{Neat Green Box!}
    \lipsum[2]
\end{commentbox}

\begin{quotebox}
    As you approach this template you get a sense that the blood and tears of many generations went into its making. A warm feeling welcomes you as you type your first words.
\end{quotebox}

\subsubsection*{Tables}

\begin{dndtable}
       \textbf{Roll}  & \textbf{Result} \\
       1-10  & Bad things \\
       10-50  & Good things \\
       50-100  & Excellent things
\end{dndtable}

\begin{dnditemtable}
       \textbf{Item}  & \textbf{Price} \\
       Basic stuff  & 10 GP \\
       Decent stuff  & 20 GP \\
       Best Stuff  & 30 GP
\end{dnditemtable}

\subsubsection*{Monster}

\begin{monsterbox}{Monster Foo}
    \textit{Small metasyntatic variable (golbinoid), neutral evil}\\
    \hline
    \basics[%
    armorclass = 12,
    hitpoints  = 16 (3d8 + 3),
    speed      = 50 ft
    ]
    \hline
    \stats[
    STR = 12 (+1),
    DEX = 14 (+2)
    ]
    \hline
    \details[%
    % If you want to use commas in these sections, enclose the
    % description in braces.
    % I'm so sorry.
    languages = {Common Lisp, Erlang},
    ]
    \hline \\[1mm]
    \begin{monsteraction}[Monster-super-powers]
        This Monster has some serious superpowers!
    \end{monsteraction}
    \monstersection{Actions}
    \begin{monsteraction}[Generate text]
        This one can generate tremendous amounts of text! Though only when it wants to.
    \end{monsteraction}

    \begin{monsteraction}[More actions]
    See, here he goes again! Yet more text.
    \end{monsteraction}
\end{monsterbox}

\subsubsection*{Spell}

\begin{spellbox}{Spell Generic}
    \spelldetails[]
    \begin{spellaction}[Materials]
    This spell doesn't really require much.
    \end{spellaction}

    \begin{spellaction}[Effect]
    Touch things and get dissaproving looks.
    \end{spellaction}
\end{spellbox}

\begin{spellbox}{Spell Druid}
    \spelldetails[%
    level   = 9,
    school  = Transmutation,
    time    = 1 action,
    range   = Touch,
    duration = Instant,
    components = {S,M},
    restrict = Druid,
    ritual = yes
    ]
    \begin{spellaction}[Materials]
    Requires tree-bark and fresh leaves.
    \end{spellaction}

    \begin{spellaction}[Effect]
    Trees will grow.
    \end{spellaction}
\end{spellbox}

\subsubsection*{Item}

\begin{itembox}{Item Generic}
    \itemdetails[]
    \begin{itemaction}[Materials]
    This item doesn't really require much.
    \end{itemaction}

    \begin{itemaction}[Effect]
    Touch things and get dissaproving looks.
    \end{itemaction}
\end{itembox}

\section{Legal \& Contact}
These rules are released under the Open Gaming License (OGL), and conform to the limirations imposed by the System Reference Document (SRD).


Following these rules will generate the life-story of your character (and their skills, proficiencies and connections) from birth right up to the point that they begin to take levels in their chosen class.
% End document
\end{multicols}
\end{document}
