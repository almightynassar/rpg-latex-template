\subsection{Kaj}
\label{sec:specie-kaj}

\includegraphics[width=\linewidth]{7e9dc5cf139ae3df9568cfd9921ba9b5}

\begin{redtable}{\linewidth}{@{}L{.35}@{}L{.65}@{}}
  \textbf{Singular} & Kaj\\
  \textbf{Plural} & Kaj\\
  \textbf{Height} & 150cm\\
  \textbf{Weight} & 55kg\\
  \textbf{Gender Ratio} & 100\% Clones\\
  \textbf{Reproduction} & Cloning\\
  \textbf{Maturity} & -\\
  \textbf{Diet} & Omnivore\\
  \textbf{Homeworld} & -\\
\end{redtable}

The Kaj are a highly intelligent species that have long ago decided to abandon the natural world and fully embrac artificial environments. The Kaj even went so far as to reject sexual reproduction and replace it with genetic engineering and cloning. While cloning has allowed the Kaj to artifically raise the intellectual quotient of their species, it has also left them susceptible to genetic diseases and viruses. This weakness to disease means that most Kaj keep a wary distance from outsiders, and zealously practice the highest hygiene standards and procedures.

They have also rejected their natural home planet, erasing the co-ordinates so that no Kaj could ever return. The Kaj generally choose to live in artificial spaces such as spacecrafts, space stations, and bubble colonies. To other species it seems that the Kaj have a driving desire to control their environment, but to most Kaj it is simply that they know better than the random processes that create the natural world.

Their is a schism of thought within Kaj society about whether or not the species as a whole should embrace the artifical completely and digitise their consciousness. Some see Digitisation as the logical continuation of the Kaj improving upon the natural, while many fear that Digitisation is a false path and would lead to the eradication of the Kaj as a species (and culture) from the galaxy. There is a whole spectrum of opinion regarding Digitisation, and this opinion usually manifests itself through how much cyberware the Kaj has. A minority of Kaj even reject the cultural acceptance of the artifical, and aim to help the Kaj rediscover the natural world.

To create a Kaj character, please refer to the \textit{\hyperref[sec:rules-creation]{Character creation section}}

\textbf{Kaj Names:}

Abupikal, Brohekano, Debimaum, Fawaroum, Frukasoiro, Gatuxund, Hanepim, Ibaxi, Nothano, Osoluga, Qitandok, Romazirash, Rumduvish, Schapruga, Strikzan, Tirwein, Xintegi, Zefinag, Zosdash, Zulaum

\textbf{Secondary Names:}

Kaj usually use the location of their cloning facility as a secondary name. Each cloning facility keeps extensive records on names to ensure that they are completely unique to the individual, such that no two Kaj from the same facility will ever have the same name across the working life of the facility.
