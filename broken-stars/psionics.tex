% =================
% Psionics
% =================

\section{Psionics}
\label{sec:psionics}

\begin{multicols}{2}

Psionics have special abilities due to the side-effects of Trans-Dimensional technology. They can manipulate matter, create fire, and maybe even alter time. Most civilians do not trust psionics for the same reasons that they do not trust AI or Ghosts, seeing them as a threat to "unaltered" organics. Many even believe that using psionic abilities will trigger a new Surge that would wipe out humanity once and for all.

A new Psionic starts off with 3 powers at the Novice level. When you wish to cast a power you must make a Psionic skill roll. You must apply all penalties noted in the table entry, including wound and fatigue penalites. You cast the power on a success (with additional benefits on a raise), but if you fail the skill roll you cancel all currently maintained powers and you are Shaken.

Some additional notes regarding powers:

\begin{itemize}

  \item \textbf{Backlash}: If you roll a 1 on the Psionic skill die (regardless of the Wild Die), your power automatically fails and you suffer 2d6 damage. You are automatically Shaken. Everyone within a Large Burst (6 diameter) centered on you also suffer 2d6 damage. On a Critical Failure (double 1's), in addition to everything else, you let out a psychic Surge that causes everyone within a Large Burst (6 diameter) to be Shaken (if they fail their Spirit roll). This can cause a wound.

  \item \textbf{Concentration}: Some powers are listed as "Concentration" and can last as long as desired. Each new power maintained in this way inflicts a -1 to cast any new powers. Thus a Shielded Psionic can keep the power going indefinitely, but suffers a -1 penalty if they then attempt to create a Blast.

  \item \textbf{Interrupting Powers}: If a character with an activated power is Shaken or suffers a wound or Fatigue level, they must make a Smarts roll to maintain all of their powers. If the roll is failed, all powers are instantly dropped. Powers stop automatically if the psionic sleeps or is rendered unconcious.

  \item \textbf{Power Preperation}: A psionic may prepare a power by concentrating for a round (no movement or other actions, and you must avoid interruptions). If successful, they ignore 2 points of penalties on all powers cast with their next action. If they do not enact any powers on their next action, the preparation is lost.

\end{itemize}

\subsection{Novice Powers}

\begin{genericsection}{Alter Light}
\textbf{Penalty: -1}\\
\textbf{Range: Smarts}\\
\textbf{Duration: Concentration}\\
Creates or negates light in a Large Burst (6 diameter) by +/-6. If the target is an opponent or item an opponent is holding, opposed by Target's agility.
\end{genericsection}

\begin{genericsection}{Attenuate}
\textbf{Penalty: -1}\\
\textbf{Range: Smarts}\\
\textbf{Duration: Concentration}\\
Sound within a Small Burst (2 diameter) is absorbed. Raise Sneak die by 1 (2 with a raise). Speaking becomes a normal action instead of Free, and you must yell to be heard. If the target is an opponent or item an opponent is holding, opposed by Target's agility.
\end{genericsection}

\begin{genericsection}{Blind}
\textbf{Penalty: -1/-2/-3}\\
\textbf{Range: 12/24/48}\\
\textbf{Duration: Instant}\\
Target must make an opposed Agility roll at -2 to avert their gaze (-4 with raise). On a failure the target is Shaken and are Blind until their next action. On a 1, they are Shaken and remain Blind until they recover from being Shaken. Blinded victims suffer -6 to all Trait rolls that require vision while they are affected and Parry is reduced to 2. You can target multiple enemines by applying -2 for Medium Burst (4 diameter), -3 for Large Burst (6 diameter). The target must see you.
\end{genericsection}

\begin{genericsection}{Burst}
\textbf{Penalty: -1}\\
\textbf{Range: Cone}\\
\textbf{Duration: Instant}\\
Targets in cone suffer 2d10 damage. This counts as a heavy weapon. Cone is 9 squares long and 3 wide.\\
\end{genericsection}

\begin{genericsection}{Confuse}
\textbf{Penalty: -1}\\
\textbf{Range: Smarts x 2}\\
\textbf{Duration: 3 rounds}\\
Target makes an opposed Smarts roll. If successful, you cause the target to lose concentration. All the target’s trait rolls are made at –2 for the duration, –4 on a raise\\
\end{genericsection}

\begin{genericsection}{Energy Bolt}
\textbf{Penalty: -1 per bold}\\
\textbf{Range: 12/24/48}\\
\textbf{Duration: Instant}\\
Up to 3 bolts at 2d6 damage. Each Bolt requires it's own psionic roll.\\
\end{genericsection}

\begin{genericsection}{Deflection}
\textbf{Penalty: -1}\\
\textbf{Range: Touch}\\
\textbf{Duration: Concentration}\\
A psionic barrier that deflects incoming attacks. -2 penalty to all fighting, shooting or other attack rolls. On a raise increases the penalty to -4. Also acts as armour against area effect weapons. Barrier only covers 1 square (usually the one you are occupying)
\end{genericsection}

\begin{genericsection}{Fear}
\textbf{Penalty: -1}\\
\textbf{Range: Smarts x 2}\\
\textbf{Duration: Instant}\\
Everyone within a Large Burst (6 diameter) must make a Fear check (at -2 with raise). Wild Cards who fail roll on the Fear table, while Extras are Panicked.
\end{genericsection}

\begin{genericsection}{Healing}
\textbf{Penalty: -2}\\
\textbf{Range: Touch}\\
\textbf{Duration: Instant}\\
Heals 1 Wound suffered within last hour, or 2 with a raise
\end{genericsection}

\begin{genericsection}{Mind Read}
\textbf{Penalty: -2}\\
\textbf{Range: Smarts}\\
\textbf{Duration: 3 rounds}\\
Opposed roll vs. target's Smarts. Allows psionic to read surface thoughts. On a raise, target is unaware of intrusion
\end{genericsection}

\begin{genericsection}{Nightvision}
\textbf{Penalty: 0}\\
\textbf{Range: Touch}\\
\textbf{Duration: Concentration}\\
Halve any darkness penalties (round down). On a raise, negate all darkness penalties up to the maximum of -6
\end{genericsection}

\begin{genericsection}{Plasma Bolt}
\textbf{Penalty: -2 per bolt}\\
\textbf{Range: 12/24/48}\\
\textbf{Duration: Instant}\\
Up to 3 bolts at 2d6 damage with armour piercing (AP 2). Each Bolt requires it's own psionic roll
\end{genericsection}

\begin{genericsection}{Psionic Manipulation}
\textbf{Penalty: -2}\\
\textbf{Range: Smarts x 2}\\
\textbf{Duration: Concentration}\\
Perform basic "tricks" with the elements. For example you can manipulate fire and heat, cool your own body, open a half-meter hole in soft earth, spray sand to blind opponent (+1 to Trick roll), and create a light breeze
\end{genericsection}

\begin{genericsection}{Restrict}
\textbf{Penalty: -1/-2}\\
\textbf{Range: Smarts}\\
\textbf{Duration: Concentration}\\
Opposed by Target's Smarts. If Target fails, you impede their nervous system and they suffer a -2 penalty to Pace, Strength and Agility checks. On a raise they are completely restrained. Affects 1 target for -1, or Medium Burst (4 diameter) for -2 (use this penalty when casting additional spells).
\end{genericsection}

\begin{genericsection}{Soothe}
\textbf{Penalty: 0}\\
\textbf{Range: Touch}\\
\textbf{Duration: Instant}\\
Removes 1 Fatigue level (2 with raise), and restores conciousness. Can remove Shaken status. Can also have a calming effect on the target
\end{genericsection}

\begin{genericsection}{Speed}
\textbf{Penalty: 0}\\
\textbf{Range: Touch}\\
\textbf{Duration: Concentration}\\
Basic Pace is doubled, and running is free action with a raise
\end{genericsection}

\begin{genericsection}{Stun}
\textbf{Penalty: -1}\\
\textbf{Range: 12/24/48}\\
\textbf{Duration: Instant}\\
All targets in Medium Burst (4 diameter) must roll a Vigor check (-2 penalty with a raise) or be Shaken\\
\end{genericsection}

\begin{genericsection}{Telekinesis (Minor)}
\textbf{Penalty: 0}\\
\textbf{Range: Smarts x 2}\\
\textbf{Duration: Concentration}\\ 
Perform a single non-combat action at range. Cannot lift objects, but can operate switches and levers
\end{genericsection}

\begin{genericsection}{Telepathy}
\textbf{Penalty: 0}\\
\textbf{Range: Smarts x 2}\\
\textbf{Duration: Concentration}\\
Allows thoughts to be transmitted, in the form of words. Once contact has been established, mental communication works in both directions. For as long as the power lasts, communication occurs as if the characters were talking face-to-face. This allows skills such as Intimidation, Persuasion, Streetwise, and Taunt to be used. More importantly, it also allows for silent communication between allies
\end{genericsection}

\begin{genericsection}{Wall Walker}
\textbf{Penalty: -1}\\
\textbf{Range: Touch}\\
\textbf{Duration: Concentration}\\
Move on any surface at half Pace, or full Pace with raise
\end{genericsection}

\subsection{Seasoned Powers}

\begin{genericsection}{Barrier}
\textbf{Penalty: -1 per section}\\
\textbf{Range: Smarts}\\
\textbf{Duration: 3 rounds}\\
Creates a solid, immobile and translucent 2m x 2m wall (1 square). This wall has a Toughness of 10. When the barrier is broken or the spell expires, the barrier dissapates. A section of the barrier is destroyed when an attack equals or exceeds it's Toughness. The barrier may be climbed at -2 penalty
\end{genericsection}

\begin{genericsection}{Blast}
\textbf{Penalty: -1/-2/-3}\\
\textbf{Range: 24/48/96}\\
\textbf{Duration: Instant}\\
Area effect power using the Medium Blast (4 diameter) and counts as a heavy weapon. A failed roll causes the blast to deviate like a launched projectile. Targets within the blast suffer 2d6 damage. Increase the penalty to -2 to do 3d6 OR a use Large Burst (6 diameter), or -3 to do both
\end{genericsection}

\begin{genericsection}{Dispel}
\textbf{Penalty: -1}\\
\textbf{Range: Smarts}\\
\textbf{Duration: Instant}\\
Negate enemy powers already in effect or to counter an enemy power as it's being used. A counter requires the psionic to be on Hold and interrupt his foe's action. Dispelling an existing power (such as barrier) requires an opposed Psionic roll
\end{genericsection}

\begin{genericsection}{Havoc}
\textbf{Penalty: -1/-2}\\
\textbf{Range: Smarts x 2}\\
\textbf{Duration: Instant}\\
With a success, use Medium Burst (4 diameter) anywhere within range. Any character in burst must make a Strength roll (at -2 on a raise). Any target that fails is knocked 2d6 squares in a random direction (roll 1d12 and read the result as a clock) and becomes prone. If the target strikes an inanimate object, they are Shaken as well. Increase penalty to -2 to use a Large Burst\\
\end{genericsection}

\begin{genericsection}{Probe}
\textbf{Penalty: -2}\\
\textbf{Range: Smarts}\\
\textbf{Duration: Instant}\\
The target makes an opposed Spirit roll opposed by his victim’s Spirit. On a success, you may get the answer to one question. The target knows he has been probed, but not necessarily by whom\\
\end{genericsection}

\begin{genericsection}{Telekinesis}
\textbf{Penalty: -2}\\
\textbf{Range: Smarts}\\
\textbf{Duration: 3 rounds}\\
Move a single object or creature. The weight you can lift is 5kg times your Spirit (or 25 kg times Spirit on a raise). Living creatures may resist with an opposed Spirit roll. On a failure, they are lifted as usual and cannot get another attempt to break free (unless they pass near something they can grab on to, which is an Strength vs Psionic roll). Affected objects are moved a number of squares up to your Smarts. Victims bashed against walls suffer Spirit+d6 damage. Dropped victims suffer normal falling damage\\
\end{genericsection}

\begin{genericsection}{Teleport}
\textbf{Penalty: -2 per 2 squares}\\
\textbf{Range: Special}\\
\textbf{Duration: Instant}\\
Disappear and reapper at a target location. This counts as movement for the round. Adjacent enemies do not get a free attack. If you teleport to somewhere you cannot see, add a -2 penalty. If the area is unknown you have never seen, use a -4 penalty. Failure means you return and are now Shaken. You can never enter a solid space. You can carry others for -1 Fatigue per passenger. Carrying more than your Fatigue means you instantly becoming Incapacitated when you reappear\\
\end{genericsection}

\subsection{Veteran Powers}

\begin{genericsection}{Greater Healing}
\textbf{Penalty: -5}\\
\textbf{Range: Touch}\\
\textbf{Duration: Instant}\\
Restores wounds more than one hour old. This otherwise acts exactly like the Healing power. It can also be used to neutralise any poison, disease, or sickness
\end{genericsection}

\begin{genericsection}{Mind Riding}
\textbf{Penalty: -2}\\
\textbf{Range: Smarts}\\
\textbf{Duration: Concentration}\\
Ability to place your mind inside someone’s else body. If the victim is an unwilling or unknowing subject, this requires an opposed Spirit roll. A mind rider gains no control over their victim, but has access to their senses and can see, hear, smell, taste, and feel everything. If the victim is injured in any way, including being Shaken by physical injury, the psionicist must make a Spirit roll or be Shaken and lose contact. A penalty of –1 applies for each wound the victim suffers. If the victim dies, the psionicist is automatically Shaken
\end{genericsection}

\subsection{Heroic Powers}

\begin{genericsection}{Divination}
\textbf{Penalty: -2}\\
\textbf{Range: Touch}\\
\textbf{Duration: 1 minute}\\
On a success, you search the fabric for the universe for an answer to one question. This question can only be answered with a Yes, No or Possibly. On a raise, the question may be answered in five words or less. The caster may take no other actions during that time, and if they are Shaken they must make a Smarts roll or the power is disrupted
\end{genericsection}

\end{multicols}
