\subsection{ATB classification}
\label{sec:sector-atb}

The \textbf{Atmospheric - Temperature - Biosphere} (ATB) rating is the standard classification system for worlds, planets and other naturally occuring stellar bodies. An ATB gives a general overview about the suitability for habitation, or at the very least an idea of what risks there are in colonising the world. The classification uses 3 letters, one for each of the categories of Atmosphere, Temperature and Biosphere.

\subsubsection{Atmosphere}

\begin{redtable}{\linewidth}{@{}L{1}@{}}
  \textbf{A (Airless)}\\
  Has no to little atmosphere to speak of\\
  \textbf{B (Breatheable)}\\
  A mix of gases that is within the breatheable range for organic beings. Since each planet has different mixtures, foriegners are always aware of the "new world stink"\\
  \textbf{C (Corrosive)}\\
  Dangerously hostile, even with conventional suits and other protective gear. The atmosphere is extremely toxic, and actively corrodes non-native materials\\
  \textbf{G (Inert Gas)}\\
  Atmosphere consists of gases that are not breatheable by organic beings. The atmosphere is otherwise not hostile or poisonous, but requires a constant source of oxygen\\
  \textbf{T (Thick)}\\
  Can be breathed with the aid of a filtration mask. Usually toxic if an organic decides to breathe the air straight over long periods of time\\
  \textbf{X (Toxic)}\\
  These worlds are much more aggressively hostile than corrosive worlds. These worlds have toxic molecules that are able to bypass suit seals during the corrosion process, meaning that they must be purified at regular intervals, reducing the oxygen supply substantially\\
\end{redtable}

\subsubsection{Temperature}

\begin{redtable}{\linewidth}{@{}L{1}@{}}
  \textbf{F (Frozen)}\\
  Average temperature that is close to absolute zero. Some are so cold that the gases have solidified\\
  \textbf{C (Cold)}\\
  These worlds are uncomfortably cold, but are survivable with suitable heavy clothing\\
  \textbf{T (Temperate)}\\
  These worlds have temperature ranges that are similar to Earth\\
  \textbf{W (Warm)}\\
  These worlds are uncomfortably warm, but are survivable. They are either desert worlds, or have thick, humid jungles\\
  \textbf{B (Burning)}\\
  Average temperatures that is close to boiling. Some are so hot that metals exist in molten form \\
  \textbf{V (Varied)}\\
  These worlds have a wide range of temperatures, either due to unique geological features or varying orbits around their star\\
\end{redtable}

\subsubsection{Biosphere}

\begin{redtable}{\linewidth}{@{}L{1}@{}}
  \textbf{R (Remnants)}\\
  The wreckage and ruins of a dead ecology\\
  \textbf{M (Microbial)}\\
  Non-sentient micro-organisms that can exist in almost all types of environments (such as slimes, bacteria, fungus). They may or may not be dangerous \\
  \textbf{Z (None)}\\
  For some reason or other, life did not evolve on this world\\
  \textbf{C (Compatible)}\\
  Substantial portion of native life is biologically compatible with human nutritional needs\\
  \textbf{I (Incompatible)}\\
  None of the native life is biologically compatible with human nutritional needs. Microbial life could potentially be highly allergenic to humans\\
  \textbf{H (Hybrid)}\\
  The native flora and fauna have been intermixed with imported species from Earth. They may or may not be compatible, but are not otherwise hostile\\
  \textbf{E (Engineered)}\\
  Are either paradise planets that have been carefully sculpted, or living forges that produce foodstuffs and minerals for trade\\
\end{redtable}
