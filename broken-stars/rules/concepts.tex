% =================
% Game Concepts
% =================
\subsection{Game Concepts}
\label{sec:rules-concepts}

While all these rules will be helpful for you getting the most out of the game, there is only one process you need to remember:

\begin{enumerate}
  \item Game master will ask you to roll some dice for a skill, attribute, ability, or other trait. Find that trait on your character sheet and grab the listed dice
  \item Roll that dice AND an extra die that is called the 'Wild die'. This die is a d6
  \item Take the highest result of the two (including Aces), and apply any bonuses or penalties
  \item Give the result to the Game Master, who will tell you if you succeed or fail. The most common target to beat is a 4
\end{enumerate}

\begin{genericsection}{Aces}
If you roll the highest number on any die, you roll that die again and add it to total (and if you roll the highest again, then you keep rolling).
\end{genericsection}

\begin{genericsection}{Asphyxiation / Atmosphere}
Organics cannot survive the Black without some form of pressurized suit. On your turn, if you have not atmosphere to breathe then you must make a Vigor check every turn. For every failure you receive a Wound as you suffer decompression.
\end{genericsection}

\begin{genericsection}{Bennies}
You get 3 Bennies at start of the game (unless you have an Edge that changes that). Use them for re-rolls and other things (such as \textbf{Soaking} damage or removing \textbf{Shaken} status). You earn more bennies for doing outstanding things during the game.
\end{genericsection}

\begin{genericsection}{Cooperative rolls}
One character will declare their action, and any able and willing companions (or systems, programs and AI) can aide the player. The character will add a +1 bonus to their results for every success and raise their companions achieved on their own (maximum bonus of +4, except for Strength). Companions must have the skill to help character in their task.
\end{genericsection}

\begin{genericsection}{Disease}
There are a wide range of diseases you can experience while travelling the Black. They are broken down into different types below for easy reference. To treat a disease (unless stated otherwise), you need the required medicine. When the medicine is available, ailments vanish in 2d6 days minus half your Vigor dice (minimum one day). Incapacitation due to disease will result in death.

You can contract a disease through \textbf{Airbourne particles} (can hold breath for 2 plus Vigor die rounds, and half that if you were not prepared), \textbf{Touch} (must make a Vigor roll immediately upon being touched), and \textbf{Bloodstream Induction} (ingested, animal bite, or infected weapon. Usually caused by being Shaken during an attack).

\begin{redtable}{\linewidth}{@{}L{1}@{}}
  \textbf{Long-Term Chronic; Majorly Debilitating}\\
  You always suffer two levels of Fatigue due to coughing fits and frequent spasms. Another Fatigue level will cause Incapacitation but not death. At the start of the session you must roll a Vigor check; on a 1 or less you are going to pass away before the session ends.\\
  \textbf{Long-Term Chronic; Minorly Debilitating}\\
  Same as above but suffer one level of Fatigue instead of two.\\
  \textbf{Short-Term; Debilitating}\\
  When you contract the disease, you suffer one level of Fatigue and you are Shaken. You can recover from being Shaken, but the Fatigue level staus for 2d6 days while the sickness works itself out.\\
  \textbf{Short-Term; Lethal}\\
  Treat this as Letal Poison. If you survive, you suffer Fatigue that lasts 2d6 hours\\
\end{redtable}
\end{genericsection}

\begin{genericsection}{Dramatic Tasks}
Some actions with deadly consequences and a time limit. You must complete 5 successful rolls to resolve the task. Most tasks come with a -2 penalty to represent the intense amount of pressure the character is under.
\end{genericsection}

\begin{genericsection}{Encumberance}
Your character's load limit is 3 x their Strength die (in kilograms). Each multiple above that limit gives a -1 penalty to Agility and Strength (\textit{do not recalculate your load limit!}) and all related skills
\end{genericsection}

\begin{genericsection}{Falling}
  You suffer 1d6+1 damage per 3 meters (2 squares), up to 10d6+10. Landing in water reduces number of dice rolled by half (rounded down).
\end{genericsection}

\begin{genericsection}{Fatigue}
Fatigue represents the stresses and weaknesses your character can suffer. It acts similarly to Wounds; once you are out of Fatigue points you become incapacitated. Every level of Fatigue is a -1 penalty to all rolls. Recovering Fatigue points is highly dependant on the source; hunger requires food, cold requires warmth, and so on. The character must resolve all sources of Fatigue to recover fully.

Some fatigue sources include:
\begin{redtable}{\linewidth}{@{}L{1}@{}}
  \textbf{Cold}\\
  Make a Vigor roll every four hours while in below freezing temperatures. -1 penalty for each -5 degrees below freezing (if wearing warm clothing). Add -2 penalty if not wearing appropriate clothing. Recovers 1 Fatigue every hour in normal temperatures.\\
  \textbf{Drowning}\\
  Make an Athletics check to avoid drowning. -2 penalty if holding something up at the same time. You roll every round while in rough water. When you are out of the water, you recover 1 Fatigue level per 5 mins. Death occurs half your Vigor die in rounds after Incapacitation.\\
  \textbf{Heat}\\
  Make a Vigor roll every four hours while in extremely hot temperatures. -1 penalty for each 5 degrees above 35 degrees Celsius (if have sufficient water). Add -2 penalty if you do not have water. Recovers 1 Fatigue every hour in normal temperatures.\\
  \textbf{Hunger/Thirst}\\
  Most organics require half a kilo of food and half a litre of water every 24 hours. If you have less food than that, you must make a Vigor roll for each 12 hour period. -2 penalty if you have had little to no food. Thirst is rolled every 6 hours. If you are incapacitated by Hunger, you die 3d6 hours later.\\
  \textbf{Radiation}\\
  You must make a Vigor roll every hour spent in low radioaction, and every minute in high radiation. Radiation Fatigue fades one level every 24 hours, or half that if you are able to scrub dust and other contaminants. On Incapacitated, you suffer a Long-Term Chronic; Minorly Debilitating disease (Radiation Sickness)\\
  \textbf{Sleep}\\
  You need a minimum of 6 hours of sleep every 24 hours. You make a Vigor roll at a culmulative -2 penalty every 12 hours you do not get the required sleep. Stimulants may add a +2 bonus to the roll. Incapacitated characters fall into sleep for 2d10 hours.
\end{redtable}
\end{genericsection}

\begin{genericsection}{Fear}
To overcome your Fears you must make a Spirit roll (some monsters add a penalty to the Spirit roll). On a failure, you are Shaken and suffer a level of Fatigue for the rest of the encounter. You only need to roll a Fear check the first time you encounter the source (or experience the source again in new and uncomfortable ways).
\end{genericsection}

\begin{genericsection}{Fire}
Anytime something flammable is hit by fire, roll a d6. On a 6 that object catches on fire. Particularly vulnerable items ignite on a 4-6. Volatile targets will catch on anything but a 1. If a person catches on fire, they need to roll every round; if they succeed again, the fire catches in intensity and causes the damage listed in the table below.

Fire in confined areas create smoke, and every round you must make a Vigor roll (+2 bonus if using a wet rag to breath through). If failed, you gain a Fatigue level.

\begin{redtable}{\linewidth}{@{}L{1}@{}}
  \textbf{Burning Weapon}\\
  +2 to damage\\
  \textbf{Spot fire}\\
  1d10\\
  \textbf{Flamethrower}\\
  2d10\\
  \textbf{Lava}\\
  3d10\\
\end{redtable}
\end{genericsection}

\begin{genericsection}{Gravity}
When you are experiencing high gravity, apply a -2 penalty to all Agility-based rolls, your pace and your jump. When you are experiencing low gravity, apply a +2 bonus to all Agility-based rols, your pace and your jump. In zero-g environments, a result of 1 on a physical trait die means you lose control and tumble (-2 to all Trait rolls). To stabilise yourself while you are tumbling, you must succeed on an Agility roll as a free action (only if you have a way to stabilise yourself).
\end{genericsection}

\begin{genericsection}{Group Rolls}
To quickly roll for a group of extras (such as henchmen), simply roll one Trait die and one Wild die. Take the highest of the two as the average for the whole group.
\end{genericsection}

\begin{genericsection}{Hacking} 
Hacking is a very complex task. It will generally require the Hacker to access a physical I/O port, which is best accessed if the Bot is incapacitated. Then the Hacker needs to open the I/O port (Hacker's Repair vs Bot's Toughness), and then break the software Firewalls (Hacker's Security opposed by the Bot's Spirit). Treat Hacking as if it was a \textbf{Dramatic Task}.
\end{genericsection}

\begin{genericsection}{Healing}
The Healing skill can be used to treat any Wound suffered within the last hour and takes about 10 minutes. A success on a Healing check removes 1 Wound, and raises will removee 2. Further raises have no effect. The healer must apply the patient's Wound penalty as well as their own Wound penalty to the Healing check. Trying to heal your own wounds will effectively double your wound penalties. If you do not have suitable medical supplies you suffer another -2 penalty. After one hour, only \textbf{Natural Healing} can remove Wounds.
\end{genericsection}

\begin{genericsection}{Initiative}
Players get one card per round and act according to the deck order, which goes from the Ace to Deuce (in case of ties the suit order is Spade, Heart, Diamond, Club). If the player gets a Joker they can act whenever they want, and the Joker gives them a +2 bonus on all Tests and +2 to all Damage
\end{genericsection}

\begin{genericsection}{Movement} 
Use the following rules to determine movement.
\begin{redtable}{\linewidth}{@{}L{1}@{}}
  \textbf{Crawling}\\
  May crawl 2 squares per turn. This counts as being prone\\
  \textbf{Crouching}\\
  May move at half Pace. You may run while crouched. Ranged attacks againts you suffer a –1 penalty\\
  \textbf{Going Prone}\\
  You may fall prone at any time during your action. Getting up costs 2 units of movement\\
  \textbf{Difficult Ground}\\
  Difficult ground such as mud, steep hills, or snow, slows characters down. Count square of movement as double\\
  \textbf{Jumping}\\
  1 square horizontally from a dead stop; 2 squares with a “run and go.” A successful Strength roll grants one extra inch of distance\\
  \textbf{Running}\\
  You may run an additional 1d6 squares during your turn, but incur a -2 running penalty to all other actions\\
\end{redtable}
\end{genericsection}

\begin{genericsection}{Natural Healing}
If you have any Wounds, you must make a Vigor roll every 5 in-game days. You remove 1 Wound level (or Incapcitated status) with a success, or improve 2 steps with a Raise. A critical failure increases your Wound level by one. You subtract wound penalties from these rolls as usual. Medical attention means that someone with the Healing skill is actively checking the patient's wounds.
\begin{redtable}{\linewidth}{@{}L{.75}@{}L{.25}@{}}
  \textbf{Condition} & \textbf{Modifier}\\
  Rough travelling & -2\\
  No medical attention & -2\\
  Poor environment & -2\\
  Medical attention (pre-industrial) & -\\
  Medical attention (industrial and beyond) & +1\\
  Medical attention (robotics and beyond) & +2\\
\end{redtable}
\end{genericsection}

\begin{genericsection}{Objects and Obstacles}
Inanimate objects have a parry of 2, no additional damage from raises on attack roll (and no aces on damage). If an attack can’t do enough damage to destroy an object, it can’t be destroyed (in combat).
\begin{redtable}{\linewidth}{@{}L{.30}@{}L{.30}@{}L{.40}@{}}
  \textbf{Object} & \textbf{Toughness} & \textbf{Damage Type}\\
  Light Door & 8 & Blunt, Cutting\\
  Heavy Door & 10 & Blunt, Cutting\\
  Lock & 8 & Blunt, Piercing\\
  Handcuffs & 12 & Blunt, Piercing, Cutting\\
  Knife, Sword & 10 & Blunt, Cutting\\
  Rope & 4 & Cutting, Piercing\\
  Shield & 10 & Blunt, Cutting\\
\end{redtable}
\end{genericsection}

\begin{genericsection}{Poison}
If ingested, effects occur automatically. If Shaken or wounded by a poisoned weapon, you must make an immediate Vigor roll (some poisons may come with a modifier). To treat a poisoned character, the medic can try a healing roll minus the strength of the poison itself. You may only attempt once per incident (another character may have a try). Fatigue levels due to poison recover 1 level per 24 hours.

\begin{redtable}{\linewidth}{@{}L{1}@{}}
  \textbf{Lethal Poison}\\
  Failure results in death in 2d6 rounds. Success results in 1 wound and Exhaustion\\
  \textbf{Venomous Poison}\\
  Failure results in death in 2d6 minutes. Success results in 1 round and Exhaustion\\
  \textbf{Paralysis Poison}\\
  Failure results in paralysis for 2d6 minutes, 2d6 rounds for success\\
  \textbf{Knockout Poison}\\
  Failure results in knocked out for 2d6 hours, 2d6 minutes for success\ 
\end{redtable}
\end{genericsection}

\begin{genericsection}{Raise}
Every 4 points over the Target Number is a Raise and can give additional benefits to your roll
\end{genericsection}

\begin{genericsection}{Size}
Most humanoids (except for the Ghoa and some Uplifted) start at the default size 0. Adjustments to this value directly affects toughness (add or subtract your size from your toughness). When attacking a creature 2 or more levels smaller than you will incur a -2 attack penalty; attacking a creature 4 or more levels larger than you will gain a +2 bonus; attacking a creature 8 or more levels larger than you will gain a +4 bonus.
\begin{redtable}{\linewidth}{@{}L{.25}@{}L{.75}@{}}
  \textbf{Size} & \textbf{Example}\\
  -2 & Cat, large rat, small dog\\
  -1 & Large dog, bobcat, small humanid\\
  0 & Humanoid, Aliens\\
  +1 & Bear Uplifted\\
  +2 & Bull, Gorilla, Horse\\
  +3 & Bear\\
  +4 & Rhino, Great White Shark\\
  +5 & Small Elephant\\
  +6 & Elephant\\
  +7 & T-Rex, Orca\\
  +8 & Dragon\\
  +9 & Blue Whale\\
  +10 & Kraken, Leviathan\\
\end{redtable}
\end{genericsection}

\begin{genericsection}{Social Conflict}
Social conflicts are broken down into three rounds of conversation. Each round the characters will roleplay their arguments and make an opposed Persuasion check (+2 for good points, -2 for faux pas). Speakers gain points for each success and raise. At the end of the third round, the character with the most points wins the conflict. If knowledge is used, the character uses the lowest trait roll between their Knowledge and Persuasion.
\begin{redtable}{\linewidth}{@{}L{.25}@{}L{.75}@{}}
  \textbf{Margin} & \textbf{Result}\\
  Tie & Issue is unsettled and no action taken until new evidence can be presented\\
  1-2 & Target is not convinced but decides it is better to be safe than sorry. Provides minimal action\\
  3-4 & Target is reasonably convinced, but will require something in return for action\\
  5+ & Target is utterly convinced, and will aide in whatever way they can\\
\end{redtable}
\end{genericsection}

\begin{genericsection}{Stealth}
Guards are either inactive or active. A success on your Stealth check will mean that you avoid inactive guards; a Failure makes them active. Active guards will then make an opposed Notice check against your Stealth result; a guard's Success means that they spot you. The last square of movement around the guard will always require an opposed Stealth/Notice check. You can move 5x you Pace when outside combat per Stealth Check. In groups, use the lowest Pace. In combat, you must make one Stealth check per round. Use the following modifiers:
\begin{redtable}{\linewidth}{@{}L{.25}@{}L{.75}@{}}
  \textbf{State} & \textbf{Modifier}\\
  Crawling & +2\\
  Running & -2\\
  Dim Light & +1\\
  Darkness & +2\\
  Pitch Black & +4\\
  Light cover & +1\\
  Medium cover & +2\\
  Heavy cover & +4\\
\end{redtable}
\end{genericsection}

\begin{genericsection}{Unskilled rolls}
If you are unskilled then your roll is a d4 with a –2 penalty
\end{genericsection}

\begin{genericsection}{Wild Die}
A d6 rolled along with normal die. Player chooses highest result (but Snakeyes is a critical fail). The Wild Die can Ace and Raise like normal dice. You only have one wild die per action (even if you use multiple Trait die, such as for Autofire)
\end{genericsection}
